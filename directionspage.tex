% XeLaTeX can use any Mac OS X font. See the setromanfont command below.
% Input to XeLaTeX is full Unicode, so Unicode characters can be typed directly into the source.

% The next lines tell TeXShop to typeset with xelatex, and to open and save the source with Unicode encoding.
%!TEX TS-program = xelatex
%!TEX encoding = UTF-8 Unicode


\documentclass[12pt]{article}
\usepackage[margin=.4in]{geometry}                % See geometry.pdf to learn the layout options. There are lots.
%\geometry{letterpaper}                   % ... or a4paper or a5paper or ... 
%\usepackage[parfill]{parskip}    % Activate to begin paragraphs with an empty line rather than an indent
\usepackage{amssymb}
\usepackage{amsmath}
\usepackage{amsthm}
\usepackage{enumerate}
\usepackage{graphicx}
\usepackage{hyperref}
\usepackage{tasks}
\usepackage{titlesec}
%\usepackage{graphicx}
%\usepackage{enumerate}
\usepackage{tabu}
\usepackage{xcolor}
%\usepackage{amssymb}
\usepackage{ifthen}
\usepackage{array}
\usepackage{dashrule}
\usepackage{xstring}

% Will Robertson's fontspec.sty can be used to simplify font choices.
% To experiment, open /Applications/Font Book to examine the fonts provided on Mac OS X,
% and change "Hoefler Text" to any of these choices.
\usepackage{fontspec,xltxtra,xunicode}
\defaultfontfeatures{Mapping=tex-text}
\setromanfont[Mapping=tex-text]{Times}
\setsansfont[Scale=MatchLowercase,Mapping=tex-text]{Gill Sans}
\setmonofont[Scale=MatchLowercase]{SimHei}


\begin{document}
    \newcommand{\jigsawtitle}{Friends are those...}
    \newcommand{\cipher}{Multiplicative}
    \newcommand{\key}{3}
    \newboolean{numbers}
    \setboolean{numbers}{true}
    \newcount \numpages
    \numpages=4
    \newcommand{\oneone}{if you don't get}
    \newcommand{\onetwo}{ }
    \newcommand{\onethree}{1003 \ 042412 \ 072413'\ 01 \ 141801}
    \newcommand{\twoone}{everything you want}
    \newcommand{\twotwo}{ }
    \newcommand{\twothree}{18231805040125101314 \ 042412 \ 08001301}
    \newcommand{\threeone}{think of the things}
    \newcommand{\threetwo}{ }
    \newcommand{\threethree}{0125101306 \ 2403 \ 012518 \ 012510131416}
    \newcommand{\fourone}{you don't get that}
    \newcommand{\fourtwo}{ }
    \newcommand{\fourthree}{042412 \ 072413'\ 01 \ 141801 \ 01250001}
    \newcommand{\fiveone}{you don't want.}
    \newcommand{\fivetwo}{ }
    \newcommand{\fivethree}{042412 \ 072413'\ 01 \ 08001301.}
    \newcommand{\sixone}{oscar wilde}
    \newcommand{\sixtwo}{ }
    \newcommand{\sixthree}{2416220005 \ 0810170718}

    \newcommand{\stripskip}{-1pt}
    \renewcommand{\arraystretch}{.8}
    \newcommand{\cell}[1]{\hbox to 3pt{\hfill \mbox{#1}\hfill} }

    \newcommand{\makerow}[1]{\cell{\StrChar{#1}{1}}&\cell{\StrChar{#1}{2}}&\cell{\StrChar{#1}{3}}& \cell{\StrChar{#1}{4}}& \cell{\StrChar{#1}{5}}&\cell{\StrChar{#1}{6}}&\cell{\StrChar{#1}{7}}&\cell{\StrChar{#1}{8}}&\cell{\StrChar{#1}{9}}&\cell{\StrChar{#1}{10}}&\cell{\StrChar{#1}{11}}&\cell{\StrChar{#1}{12}}&\cell{\StrChar{#1}{13}}&\cell{\StrChar{#1}{14}}&\cell{\StrChar{#1}{15}}&\cell{\StrChar{#1}{16}}&\cell{\StrChar{#1}{17}}&\cell{\StrChar{#1}{18}}&\cell{\StrChar{#1}{19}}}
       \newcommand{\makenumbersrow}[1]{\scriptsize\cell{\StrMid{#1}{1}{2}}&\scriptsize\cell{\StrMid{#1}{3}{4}}&\scriptsize\cell{\StrMid{#1}{5}{6}}& \scriptsize\cell{\StrMid{#1}{7}{8}}& \scriptsize\cell{\StrMid{#1}{9}{10}}&\scriptsize\cell{\StrMid{#1}{11}{12}}&\scriptsize\cell{\StrMid{#1}{13}{14}}&\scriptsize\cell{\StrMid{#1}{15}{16}}&\scriptsize\cell{\StrMid{#1}{17}{18}}&\scriptsize\cell{\StrMid{#1}{19}{20}}&\scriptsize\cell{\StrMid{#1}{21}{22}}&\scriptsize\cell{\StrMid{#1}{23}{24}}&\scriptsize\cell{\StrMid{#1}{25}{26}}&\scriptsize\cell{\StrMid{#1}{27}{28}}&\scriptsize\cell{\StrMid{#1}{29}{30}}&\scriptsize\cell{\StrMid{#1}{31}{32}}&\scriptsize\cell{\StrMid{#1}{33}{34}}&\scriptsize\cell{\StrMid{#1}{35}{36}}&\scriptsize\cell{\StrMid{#1}{37}{38}}}
     \newcommand{\makestrip}[3]{
%		{\small{\cipher \hfill\key}}	

		\begin{tabu}{ c c c c c c c c c c c c c c c c c c c  }
			\rowfont{\color{\anscolor}}
    			\makerow{#1}\\[-2pt]
    			\ifthenelse{\boolean{numbers}}{\makenumbersrow{#3}}{\makerow{#3}}\\
		\end{tabu}	
		 }
    \newcount \pagenumber
    \includegraphics[scale = 3]{logo_small}\vskip -1in
    \hfill 
    			\begin{minipage}{.4\textwidth}{ 
			{\bf\LARGE {\jigsawtitle}}\\[2pt] 
			{\large{\cipher{ Cipher,} Key = \key}} }
			\end{minipage}

\vskip .6in

{\bf{Directions for setting up your Jigsaw Message}} \\[2pt]
\begin{minipage}{.05\textwidth}{\hspace{1pt}}\end{minipage}
    	\begin{minipage}[c]{.9\textwidth}
    	\begin{enumerate}[\color{blue} 1.]
	  \setlength{\itemsep}{0pt}
  	\setlength{\parskip}{0pt}
    	\item {\color{blue}{Print}} one copy of the jigsaw strips for each group of students.
	\item {\color{blue}{Prepare a packet}}  for each group:
		\begin{enumerate}[a.]
		  \setlength{\itemsep}{0pt}
  		  \setlength{\parskip}{0pt}
			\item Cut out the strips and arrange the strips into random order.
			\item Place the strips into an envelope or simply staple them together.
		\end{enumerate}
	\item {\color{blue}{Divide students into groups. }} Divide students into groups. Every student should have at least one strip to decrypt, so plan for fewer students in a
group than strips in your message.
	\item {\color{blue}{Begin the game}}  by distributing the packets to the groups.
	\item {\color{blue}{Game play: }} Each groups receives an envelope of strips. They must decrypt the message and put the pieces into the right order. The first team to discover the message is the winner. To play the game as a relay race, see the instructions for Cipher Relay, available in the Jigsaw generator section of the CryptoClub website or in the CryptoClub Leader Manual.
    \end{enumerate}
    \end{minipage}
 \vskip .2in   
{\bf{Tips:}}\\[2pt]
\begin{minipage}{.05\textwidth}{\hspace{1pt}}\end{minipage}
    	\begin{minipage}[c]{.9\textwidth}
Teamwork makes the decryption go faster. Each team player gets one strip to decipher and add to the puzzle. Faster players can do longer lines, do more lines and/or share information with those still deciphering. If the cipher is a simple substitution, once a single letter is deciphered, the whole group can use that information.
    \end{minipage}
 \vskip .2in   
\newcommand{\anscolor}{red}
{\bf{Message Text}}\\
\pagenumber=1
{\hspace{1em}}
 
    \fbox{ 
    \begin{minipage}{220pt}
    \setlength{\tabcolsep}{4pt}
    \vskip .3cm
    \texttt{\vbox {
    \makestrip{\oneone}{\onetwo}{\onethree}
   }
    \vbox { 
    \vskip \stripskip
    \makestrip{\twoone}{\twotwo}{\twothree}
    }
    \vbox {
        \vskip \stripskip 
      \makestrip{\threeone}{\threetwo}{\threethree}
    }
       \vbox { 
           \vskip \stripskip
      \makestrip{\fourone}{\fourtwo}{\fourthree}
    }
    \vbox { 
        \vskip \stripskip
      \makestrip{\fiveone}{\fivetwo}{\fivethree}
    }
       \vbox { 
           \vskip \stripskip
      \makestrip{\sixone}{\sixtwo}{\sixthree}
    }
    \vskip 1pt}
 \end{minipage}
 } 
 \ifnum\pagenumber=\numpages {\end{document}}{\pagenumber=2}\fi
\renewcommand{\oneone}{if you don't get}
\renewcommand{\onetwo}{ }
\renewcommand{\onethree}{1003 \ 042412 \ 072413'\ 01 \ 141801}
\renewcommand{\twoone}{everything you want}
\renewcommand{\twotwo}{ }
\renewcommand{\twothree}{18231805040125101314 \ 042412 \ 08001301}
\renewcommand{\threeone}{think of the things}
\renewcommand{\threetwo}{ }
\renewcommand{\threethree}{0125101306 \ 2403 \ 012518 \ 012510131416}
\renewcommand{\fourone}{you don't get that}
\renewcommand{\fourtwo}{ }
\renewcommand{\fourthree}{042412 \ 072413'\ 01 \ 141801 \ 01250001}
\renewcommand{\fiveone}{you don't want.}
\renewcommand{\fivetwo}{ }
\renewcommand{\fivethree}{042412 \ 072413'\ 01 \ 08001301.}
\renewcommand{\sixone}{oscar wilde}
\renewcommand{\sixtwo}{ }
\renewcommand{\sixthree}{2416220005 \ 0810170718}
    \hfil\fbox{ 
    \begin{minipage}{220pt}
    \setlength{\tabcolsep}{4pt}
    \vskip .3cm
    \texttt{\vbox {
    \makestrip{\oneone}{\onetwo}{\onethree}
   }
    \vbox { 
    \vskip \stripskip
    \makestrip{\twoone}{\twotwo}{\twothree}
    }
    \vbox {
        \vskip \stripskip 
      \makestrip{\threeone}{\threetwo}{\threethree}
    }
       \vbox { 
           \vskip \stripskip
      \makestrip{\fourone}{\fourtwo}{\fourthree}
    }
    \vbox { 
        \vskip \stripskip
      \makestrip{\fiveone}{\fivetwo}{\fivethree}
    }
       \vbox { 
           \vskip \stripskip
      \makestrip{\sixone}{\sixtwo}{\sixthree}
    }
    \vskip 1pt}
 \end{minipage}
 }
 \vskip 1pt
\ifnum\pagenumber=\numpages {\end{document}}{}\fi
\pagenumber=3
\renewcommand{\oneone}{this is the newer}
\renewcommand{\onetwo}{0123456789012345678}
\renewcommand{\onethree}{01020304050607080910111213141516171819}
\renewcommand{\twoone}{this is the two}
\renewcommand{\twotwo}{THIS IS THE TWO}
\renewcommand{\twothree}{THIS IS THE SECOND STRIP}
\renewcommand{\threeone}{this is the three}
\renewcommand{\threetwo}{THIS IS THE THREE}
\renewcommand{\threethree}{THIS IS THE THIRD STRIP}
\renewcommand{\fourone}{this is the four}
\renewcommand{\fourtwo}{THIS IS THE FOUR}
\renewcommand{\fourthree}{THIS IS THE FOURTH STRIP}
\renewcommand{\fiveone}{this is the five}
\renewcommand{\fivetwo}{THIS IS THE FIVE}
\renewcommand{\fivethree}{THIS IS THE FIFTH STRIP}
\renewcommand{\sixone}{ }
\renewcommand{\sixtwo}{ }
\renewcommand{\sixthree}{ }
{\hspace{1em}}\\

    \fbox{ 
    \begin{minipage}{220pt}
    \setlength{\tabcolsep}{4pt}
    \vskip .3cm
    \texttt{\vbox {
    \makestrip{\oneone}{\onetwo}{\onethree}
   }
    \vbox { 
    \vskip \stripskip
    \makestrip{\twoone}{\twotwo}{\twothree}
    }
    \vbox {
        \vskip \stripskip 
      \makestrip{\threeone}{\threetwo}{\threethree}
    }
       \vbox { 
           \vskip \stripskip
      \makestrip{\fourone}{\fourtwo}{\fourthree}
    }
    \vbox { 
        \vskip \stripskip
      \makestrip{\fiveone}{\fivetwo}{\fivethree}
    }
       \vbox { 
           \vskip \stripskip
      \makestrip{\sixone}{\sixtwo}{\sixthree}
    }
    \vskip 1pt}
 \end{minipage}
 }
\ifnum\pagenumber=\numpages {\end{document}}{}\fi
\pagenumber=4
\renewcommand{\oneone}{if you don't get}
\renewcommand{\onetwo}{ }
\renewcommand{\onethree}{1003 \ 042412 \ 072413'\ 01 \ 141801}
\renewcommand{\twoone}{everything you want}
\renewcommand{\twotwo}{ }
\renewcommand{\twothree}{18231805040125101314 \ 042412 \ 08001301}
\renewcommand{\threeone}{think of the things}
\renewcommand{\threetwo}{ }
\renewcommand{\threethree}{0125101306 \ 2403 \ 012518 \ 012510131416}
\renewcommand{\fourone}{you don't get that}
\renewcommand{\fourtwo}{ }
\renewcommand{\fourthree}{042412 \ 072413'\ 01 \ 141801 \ 01250001}
\renewcommand{\fiveone}{you don't want.}
\renewcommand{\fivetwo}{ }
\renewcommand{\fivethree}{042412 \ 072413'\ 01 \ 08001301.}
\renewcommand{\sixone}{oscar wilde}
\renewcommand{\sixtwo}{ }
\renewcommand{\sixthree}{2416220005 \ 0810170718}
\hfil
    \fbox{ 
    \begin{minipage}{220pt}
    \setlength{\tabcolsep}{4pt}
    \vskip .3cm
    \texttt{\vbox {
    \makestrip{\oneone}{\onetwo}{\onethree}
   }
    \vbox { 
    \vskip \stripskip
    \makestrip{\twoone}{\twotwo}{\twothree}
    }
    \vbox {
        \vskip \stripskip 
      \makestrip{\threeone}{\threetwo}{\threethree}
    }
       \vbox { 
           \vskip \stripskip
      \makestrip{\fourone}{\fourtwo}{\fourthree}
    }
    \vbox { 
        \vskip \stripskip
      \makestrip{\fiveone}{\fivetwo}{\fivethree}
    }
       \vbox { 
           \vskip \stripskip
      \makestrip{\sixone}{\sixtwo}{\sixthree}
    }
    \vskip 1pt}
 \end{minipage}
 }

\end{document}  
%  {\footnotesize{\it{This is the footer}}}
% For many users, the previous commands will be enough.
% If you want to directly input Unicode, add an Input Menu or Keyboard to the menu bar 
% using the International Panel in System Preferences.
% Unicode must be typeset using a font containing the appropriate characters.
% Remove the comment signs below for examples.

% \newfontfamily{\A}{Geeza Pro}
% \newfontfamily{\H}[Scale=0.9]{Lucida Grande}
% \newfontfamily{\J}[Scale=0.85]{Osaka}

% Here are some multilingual Unicode fonts: this is Arabic text: {\A السلام عليكم}, this is Hebrew: {\H שלום}, 
% and here's some Japanese: {\J 今日は}.

